%%%%%%%%%%%%%%%%%%%%%%%%%%%%%%%%%%%%%%%%%%%%%%%%%%%%%%%%%%%%%%%%%%
%%%%%%%% ICML 2017 EXAMPLE LATEX SUBMISSION FILE %%%%%%%%%%%%%%%%%
%%%%%%%%%%%%%%%%%%%%%%%%%%%%%%%%%%%%%%%%%%%%%%%%%%%%%%%%%%%%%%%%%%

% Use the following line _only_ if you're still using LaTeX 2.09.
%\documentstyle[icml2017,epsf,natbib]{article}
% If you rely on Latex2e packages, like most moden people use this:
\documentclass{article}

% use Times
\usepackage{times}
% For figures
\usepackage{graphicx} % more modern
%\usepackage{epsfig} % less modern
\usepackage{subfigure} 

% For citations
\usepackage{natbib}

% For algorithms
\usepackage{algorithm}
\usepackage{algorithmic}

% As of 2011, we use the hyperref package to produce hyperlinks in the
% resulting PDF.  If this breaks your system, please commend out the
% following usepackage line and replace \usepackage{icml2017} with
% \usepackage[nohyperref]{icml2017} above.
\usepackage{hyperref}

% Packages hyperref and algorithmic misbehave sometimes.  We can fix
% this with the following command.
\newcommand{\theHalgorithm}{\arabic{algorithm}}

% Employ the following version of the ``usepackage'' statement for
% submitting the draft version of the paper for review.  This will set
% the note in the first column to ``Under review.  Do not distribute.''
%\usepackage{icml2017} 

% Employ this version of the ``usepackage'' statement after the paper has
% been accepted, when creating the final version.  This will set the
% note in the first column to ``Proceedings of the...''
\usepackage[accepted]{icml2017}


% The \icmltitle you define below is probably too long as a header.
% Therefore, a short form for the running title is supplied here:
\icmltitlerunning{Applying Unsupervised Learning to Constitution Based Data}

\begin{document} 

\twocolumn[
	\icmltitle{Applying Unsupervised Learning to Constitution Based Data For Geographical and Temporal Grouping}

% It is OKAY to include author information, even for blind
% submissions: the style file will automatically remove it for you
% unless you've provided the [accepted] option to the icml2017
% package.

% list of affiliations. the first argument should be a (short)
% identifier you will use later to specify author affiliations
% Academic affiliations should list Department, University, City, Region, Country
% Industry affiliations should list Company, City, Region, Country

% you can specify symbols, otherwise they are numbered in order
% ideally, you should not use this facility. affiliations will be numbered
% in order of appearance and this is the preferred way.
\icmlsetsymbol{equal}{*}

\begin{icmlauthorlist}
	\icmlauthor{Tyler Hartwig}{bu}
\end{icmlauthorlist}

\icmlaffiliation{bu}{Baylor University, Waco, Texas}


% You may provide any keywords that you 
% find helpful for describing your paper; these are used to populate 
% the "keywords" metadata in the PDF but will not be shown in the document
\icmlkeywords{}

\vskip 0.3in
]

% this must go after the closing bracket ] following \twocolumn[ ...

% This command actually creates the footnote in the first column
% listing the affiliations and the copyright notice.
% The command takes one argument, which is text to display at the start of the footnote.
% The \icmlEqualContribution command is standard text for equal contribution.
% Remove it (just {}) if you do not need this facility.

%\printAffiliationsAndNotice{}  % leave blank if no need to mention equal contribution
\printAffiliationsAndNotice{\icmlEqualContribution} % otherwise use the standard text.
%\footnotetext{hi}

\begin{abstract} 
	Machine Learning is applied to a database of various variables applied to various constitutions from the various countries. In particular, unsupervised learning used to naturally group these constitutions by geographical region, and by the time frame in which they are written. Due to a large number of variables, several tactics were employed to reduce dimensionality and make the learning problem feasible. 
\end{abstract} 

\section{Introduction}
Dr. Curt Nichols of Baylor University posses a database of various (mostly binary) variables on constitutions from nations around the world. Dr. Nichols proposed the idea of grouping the countries based on a subset of these variables, with an end goal of naturally grouping the data according to geographical region and the time period in which the constitution was written. Realistically when constitutions are written, they are based on other constitutions. With this in mind, it is likely a particular country will base their constitution on the constitutions from the countries surrounding them, or recently created constitutions. This is the main motivation for looking for geographical and temporal groupings in this paper.

The database itself has over 1000 variables for each country (including geographical and temporal information) making this problem practically infeasible if one desires to consider all present variables. Many tactics are used to combat this, such as: reducing the number of variables used, principle component analysis, and uniquely defined difference functions which define a smaller set of dimensions. Additionally, no geographical or temporal variables are used, as this would remove the statistical implications of the countries naturally being grouped by region or time period.

This problem very naturally lends itself to hierarchical clustering, and is the subset of unsupervised learning which will be used in the remainder of the paper. This kind of grouping is natural, as it can easily depict the closeness of one country to another, both in time or space. 

\section{Related Research}
Greg Ver Steeg and Aram Galstyan present a method for finding the most informative hierarchical clustering for high dimensional data \cite{HD-Clustering}. The paper illustrates how the variables can be estimated using multivariate probability theory in order to obtain a similarity measurement. This leads way to a neural network which can then maximize this similarity. Additionally a computationally feasibly method is presented for using this similarity definition for hierarchical clustering. 

\section{Experiment Setup}
One of the main focuses of this experiment is to reduce the dimensionality of the problem into a useful size. This is done in a few ways to start. According to Dr. Nichols, our domain expert, the variables of particular interest are the Executive, Legislative, and Judiciary branch variables, along with those pertaining to Criminal Procedures and Rights. Many variables are removed as well, particularly those without numeric values. This initially cuts the data to be about 700 variables. This is clearly still an incredibly large number of dimensions.

The data is initially clustered off of this data in euclidean distance, which unsurprisingly does not yield acceptable results. This is not surprising, as the data is generally binary, however these variables often take on large values to indicate the absence of the variable (rather than true or false). It is obvious that a better distance metric is necessary to this problem. One of the first metrics used is a Hamming distance between each data point. This is a much more natural evaluation of binary variables. How this works specifically is each variable for each country is compared against each other country, when the variables are the same, their ``similarity'' is increased. This allows the data to again be clustered, yielding more favorable results geographically. 

One other method is used (currently) which is essentially redefine the variables used in this experiment. Dr. Nichols also has interest in how similar each constitution is to that of several ``base'' constitutions. Particularly, The United States, British and Austrian constitutions serve as a standard to compare against. A new coordinate system can then be defined, in which each axis is a ``base'' constitution, and each value is the similarity of a given constitution to each axis. This measurement does not give outstanding results either, and the domain expert will not be contacted to further this project.

% In the unusual situation where you want a paper to appear in the
% references without citing it in the main text, use \nocite
\nocite{HD-Clustering}

\bibliography{constitution_project}
\bibliographystyle{icml2017}

\end{document} 


% This document was modified from the file originally made available by
% Pat Langley and Andrea Danyluk for ICML-2K. This version was
% created by Lise Getoor and Tobias Scheffer, it was slightly modified  
% from the 2010 version by Thorsten Joachims & Johannes Fuernkranz, 
% slightly modified from the 2009 version by Kiri Wagstaff and 
% Sam Roweis's 2008 version, which is slightly modified from 
% Prasad Tadepalli's 2007 version which is a lightly 
% changed version of the previous year's version by Andrew Moore, 
% which was in turn edited from those of Kristian Kersting and 
% Codrina Lauth. Alex Smola contributed to the algorithmic style files.  
